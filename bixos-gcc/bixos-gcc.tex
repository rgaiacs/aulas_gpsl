\documentclass{beamer}

\input{../lib/estilo.tex}
\input{../lib/titulo.tex}
% configurações do lstlisting para simular um shell

\usepackage{listings}
\usepackage{color}
\usepackage[defaultmono]{droidmono}

\lstset{
    basicstyle=\ttfamily\color{white},
    language=bash,
    stepnumber=0, % sem númvero de linhas
    %backgroundcolor=\color{black},
    tabsize=4,
    captionpos=b,
    keywordstyle=\color{green},
    showspaces=false,
    showstringspaces=false,
    inputencoding=utf8,
    extendedchars=true,
    numberstyle=\scriptsize,
    literate={á}{{\'a}}1
}

% coloca um comando de shell com a formatação correta
% o argumento opcional é o texto do prompt
\newcommand{\shellcommand}[2]{
    \textcolor{white}{#1 \ #2}
}

\newcommand{\usercmd}[2][]{
    \shellcommand{#1\$}{#2}
}

\newcommand{\rootcmd}[1]{
    \shellcommand{\#}{#1}
}

% comando para inserir um comentário com a formatação apropriada
\newcommand{\comment}[1]{
    \textcolor{gray}{\# \ {\em #1 }}
}

\newsavebox{\shellbox}
\newenvironment{shell}
{
    \global\begin{lrbox}{\shellbox}
    \begin{minipage}[c]{0.8\textwidth}
    \begin{tt}
}
{
    \nolinebreak[4]
    \end{tt}
    \end{minipage}
    \end{lrbox}
    \colorbox{black}{\usebox{\shellbox}}
}


\usepackage{xspace}

\begin{document}

\newcommand{\software}{\emph{software}\xspace}
\newcommand{\Software}{\emph{Software}\xspace}
\newcommand{\softwarelivre}{\Software Livre\xspace}
\newcommand{\opensource}{\emph{Open-Source}\xspace}
\gpsltitle{Aula de Compilação}

\begin{frame}{Objetivos}
  \begin{itemize}
    \item Conhecer as bibliotecas básicas de C.
    \item Compilar o ``Hello World''.
    \item Compilar utilizando Makefile.
    \item Conhecer algumas flags do gcc.
    \item Resolver alguns erros e avisos.
  \end{itemize}
\end{frame}


\begin{frame}{glibc}
  glibc significa GNU C Library e é a biblioteca C mais utilizada na
  atualidade.

  Informações sobre a glibc encontram-se no formato de man pages,
  principalmente na terceira seção .

  \begin{center}
    \begin{shell}
      \usercmd{man 3 printf}
    \end{shell}
  \end{center}
\end{frame}

\begin{frame}[fragile]{hello.c}
  \lstinputlisting[basicstyle=\color{black},
      keywordstyle=\color{black}, language=C]{hello.c}
\end{frame}

\begin{frame}{Compilação}
  \begin{center}
    \begin{shell}
      \usercmd{cat hello.c}
      \usercmd{gcc -o hello hello.c}
    \end{shell}
  \end{center}
\end{frame}

\begin{frame}{Execução}
  \begin{center}
    \begin{shell}
      \usercmd{./hello}{hello world.}
    \end{shell}
  \end{center}
\end{frame}

\begin{frame}[fragile]{Makefile}
  \lstinputlisting[basicstyle=\color{black},
      keywordstyle=\color{black}, language=make]{helloMakefile}
\end{frame}

\end{document}
