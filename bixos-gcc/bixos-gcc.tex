\documentclass{beamer}

\input{../lib/estilo.tex}
\input{../lib/titulo.tex}
\input{../lib/shell.tex}

\usepackage{xspace}

\begin{document}

\newcommand{\software}{\emph{software}\xspace}
\newcommand{\Software}{\emph{Software}\xspace}
\newcommand{\softwarelivre}{\Software Livre\xspace}
\newcommand{\opensource}{\emph{Open-Source}\xspace}
\gpsltitle{Aula de Compilação}

\begin{frame}{Objetivos}
  \begin{itemize}
    \item Conhecer as bibliotecas básicas de C.
    \item Compilar o ``Hello World''.
    \item Compilar utilizando Makefile.
    \item Conhecer algumas flags do gcc.
    \item Resolver alguns erros e avisos.
  \end{itemize}
\end{frame}


\begin{frame}{glibc}
  glibc significa GNU C Library e é a biblioteca C mais utilizada na
  atualidade.

  Informações sobre a glibc encontram-se no formato de man pages,
  principalmente na terceira seção .

  \begin{center}
    \begin{shell}
      \usercmd{man 3 printf}
    \end{shell}
  \end{center}
\end{frame}

\begin{frame}[fragile]{hello.c}
  \lstinputlisting[basicstyle=\color{black},
      keywordstyle=\color{black}, language=C]{hello.c}
\end{frame}

\begin{frame}{Compilação}
  \begin{center}
    \begin{shell}
      \usercmd{cat hello.c}
      \usercmd{gcc -o hello hello.c}
    \end{shell}
  \end{center}
\end{frame}

\begin{frame}{Execução}
  \begin{center}
    \begin{shell}
      \usercmd{./hello}{hello world.}
    \end{shell}
  \end{center}
\end{frame}

\begin{frame}[fragile]{Makefile}
  \lstinputlisting[basicstyle=\color{black},
      keywordstyle=\color{black}, language=make]{helloMakefile}
\end{frame}

\end{document}
